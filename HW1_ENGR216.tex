\documentclass[letterpaper,11pt]{article}
\usepackage{amsmath, amssymb}
\usepackage{geometry}
\geometry{a4paper, margin=1in}
\usepackage{mathrsfs}
\usepackage{mathdots}
\DeclareMathOperator{\sech}{sech}
\date{\vspace{-2cm}} % Removed the date
\usepackage{latexsym}
\usepackage[empty]{fullpage}
\usepackage{titlesec}
\usepackage{marvosym}
\usepackage[usenames,dvipsnames]{color}
\usepackage{verbatim}
\usepackage{enumitem}
\usepackage{graphicx}
\usepackage[hidelinks]{hyperref}
\usepackage{fancyhdr}
\usepackage[english]{babel}
\usepackage{tabularx}
\input{glyphtounicode}
\usepackage{silence}
\WarningsOff*
\ErrorsOff*
\begin{document}
\pagestyle{plain}
\setlength{\parskip}{1em}
\setlength{\parindent}{0em}
\newcommand{\resumeSubheading}[6]{
  \vspace{-2pt}\item
    \begin{tabular*}{0.97\textwidth}[t]{l@{\extracolsep{\fill}}r}
      \textbf{#1} & #2 \\
       #3 & #4\\
      #5 & #6 \\
    \end{tabular*}\vspace{-7pt}
}
\resumeSubheading{HW 1: Descriptive Statistics and Measurement Error}{Evan Kniffen}{19 January 2025}{435003600}{Total Pages: 2}{ENGR 216 - 525}
\\
\\
\\
\textbf{Problem 1:}

\textbf{Given:} \linebreak
Concrete\_Data.xlsx: An Excel sheet with 1031 inputted rows of concrete data, with measurements from which we will be calculating our statistics. The file Concrete\_Data.xlsx contains data that was collected through
research by I-Cheng Yeh [1].

\textbf{Find:}\linebreak
a) The mean, median, and mode of the fly ash component. \\
b) The range, variance, and standard deviation of the coarse aggregate component.\\
c) The mean and standard error of the cement component.\\
d) The mean and standard error of the age.

\textbf{Diagram:}\linebreak
\begin{figure}[h!]
    \centering
    \includegraphics[width=0.25\linewidth]{640px-Bloczek_betonowy.jpg}
    \caption{Concrete [2]}
\end{figure}

\textbf{Theory:}\linebreak
Population Variance: \(\displaystyle\sigma^2 = \frac1n\sum_{i=1}^n(x_i-\mu)^2\)

Population Standard Deviation = \(\sqrt{\sigma^2} = \sigma = \sqrt{\displaystyle \frac1n\sum_{i=1}^n(x_i-\mu)^2}\)

Standard Error=\(\frac{\sigma}{\sqrt{n}}\)

\textbf{Assumptions:}\linebreak
We can assume that the population equations are applicable given the large sample size. 

\textbf{Solutions:} \emph{Below each numerical solution set is the Excel code used in Concrete\_Data.xlsx to obtain that solution respectively.}\\
\textbf{a)} Fly Ash Component:\\
Mean: \framebox{54.19 \(\text{kg}\)/\(\text{m}^3\)}\\
Median: \framebox{0.0 \(\text{kg}\)/\(\text{m}^3\)}\\
Mode: \framebox{0.0 \(\text{kg}\)/\(\text{m}^3\)}\\
\texttt{=AVERAGE(C2:C1031)}\\
\texttt{=MEDIAN(C2:C1031)}\\
\texttt{=MODE.SNGL(C2:C1031)}

\textbf{b)} Coarse Aggregate Component:\\
Range: \framebox{344.0 \(\text{kg}\)/\(\text{m}^3\)}\\
Variance: \framebox{6045.66 \(\text{kg}^2\)/\(\text{m}^6\)}\\
Standard Deviation: \framebox{77.75 \(\text{kg}\)/\(\text{m}^3\)}\\
\texttt{=MAX(F2:F1031) - MIN(F2:F1031)}\\
\texttt{=VAR.S(F2:F1031)}\\
\texttt{=STDEV.S(F2:F1031)}

\textbf{c)} Cement Component:\\
Mean: \framebox{281.17 \(\text{kg}\)/\(\text{m}^3\)}\\
Standard Error: \framebox{3.26 \(\text{kg}\)/\(\text{m}^3\)}\\
\texttt{=AVERAGE(A2:A1031)}\\
\texttt{=STDEV.S(A2:A1031) / SQRT(COUNT(A2:A1031))}

\textbf{d)} Age:\\
Mean: \framebox{45.66 days}\\
Standard Error: \framebox{1.97 days}\\
\texttt{=AVERAGE(H2:H1031)}\\
\texttt{=STDEV.S(H2:H1031) / SQRT(COUNT(H2:H1031))}

These solutions all make sense given a quick look at the trends within the Excel sheet; most of the columns with a significant amount of zeroes have a mode and/or median of zero (this is true for more than just our Part A calculation). Furthermore, the age of each sample has generally lower values when compared to the trends seen in the other columns, leading to a lower mean. 

[1] I-Cheng Yeh, “Modeling of strength of high performance concrete using artificial neural networks,”
Cement and Concrete Research, Vol. 28, No. 12, pp. 1797-1808 (1998).

[2] Ablazejo, "Bloczek betonowy," Wikimedia Commons, February 14, 2014.

\end{document}